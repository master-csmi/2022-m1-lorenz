%% Requires compilation with XeLaTeX or LuaLaTeX
\documentclass[10pt,xcolor={table,dvipsnames},t]{beamer}
\usetheme{diapo}
\usepackage{amsmath}

\title[Your Short Title]{Lorenz System Project}
\subtitle{Presentation (version 0)}
\author[name]{AYDOGDU Melissa, LECOURTIER Frédérique}
\institute{\large Strasbourg University}
\date{05 april 2022}

\useoutertheme{miniframes}

\begin{document}
	
	\begin{frame}
		\titlepage
	\end{frame}
	
	\AtBeginSection[]{
		\begin{frame}
			\vfill
			\centering
			\begin{beamercolorbox}[sep=5pt,shadow=true,rounded=true]{subtitle}
				\usebeamerfont{title}\insertsectionhead\par%
			\end{beamercolorbox}
			\vfill
		\end{frame}
	}
	
	\begin{frame}{Goals of the project}
		
		\pgfimage[height=5.5cm,width=11cm]{images/gantt}
		
	\end{frame}
	
	
	\section{Introduction}
	
	\begin{frame}{Lorenz system}
		
		The system :
		$$\left\{\begin{aligned} 
			x'&=\sigma(y-x) \\
			y'&=x(r-z)-y \\
			z'&=xy-bz
		\end{aligned}\right.$$
		
		\begin{enumerate}
			\item Non-linear - two nonlinearities : $xy$ and $xz$
			\item symmetry - invarient under $(x,y)\rightarrow(-x,-y) $
			\item 3 fixed points - $(0,0,0) ,(\sqrt{b(r-1)},\sqrt{b(r-1)},r-1)$ and $(-\sqrt{b(r-1)},-\sqrt{b(r-1)},r-1)$
		\end{enumerate}
		
	\end{frame}
	
	\section{Some interesting properties}
	
	\begin{frame}{Lorenz Attractor}
		
		\begin{figure}[h]
			\begin{minipage}[c]{.46\linewidth}
				\centering
				\includegraphics[width=\textwidth]{images/butterfly.jpg}
			\end{minipage}
			\hfill
			\begin{minipage}[c]{.46\linewidth}
				\centering
				\includegraphics[width=\textwidth]{images/butterfly3D.jpg}
			\end{minipage}
		\end{figure}
		
	\end{frame}
	
	\begin{frame}[allowframebreaks]{Chaos theory}
		
		\begin{center}
			\begin{minipage}[c]{.99\linewidth}
				\centering
				\includegraphics[width=0.7\textwidth]{images/chaos1.png}
			\end{minipage}
		\end{center}
		
		\newpage
		
		\begin{center}
			\begin{minipage}[c]{.99\linewidth}
				\centering
				\includegraphics[width=0.7\textwidth]{images/chaos2.png}
			\end{minipage}
		\end{center}
		
	\end{frame}
	
	\section{Numerical resolution with different methods}
	
	\begin{frame}[allowframebreaks]{Methods used to solve Lorenz system}
		
		$f : [0; T] \times \mathbb{R}^n \rightarrow \mathbb{R}^n$, \; a continuous function.
		
		For $X_0\in \mathbb{R}^n$, we're searching $X\in C^1([0,T],\mathbb{R}^n)$ a solution for :
		
		$$\left\{\begin{aligned}
			X'&=f(t,X) \\
			X(0)&=X_0
		\end{aligned}\right.$$
		
		To solve the Lorenz problem we will have:
		
		$$X'=\begin{pmatrix}
			x' \\
			y' \\
			z'
		\end{pmatrix}, \quad X=\begin{pmatrix}
			x \\
			y \\
			z
		\end{pmatrix} \quad et \quad f(t,X)=\begin{pmatrix}
			\sigma(y-x) \\
			x(r-z)-y \\
			xy-bz
		\end{pmatrix}$$
		
		\newpage
		
		\begin{itemize}
			\item Discretization : \\
			\quad \includegraphics[width=0.4\textwidth]{images/discretization.jpg} \\ 
			Let \quad $t_n=n\Delta t$ \quad with \quad $\Delta t=T/N$ \quad and \quad $X_n=X(t_n)$ \\ \; \\
			\item Explicit Euler :
			$$X_{n+1}=X_n+\Delta t f(t_n,X_n)$$
			\item Implicit Euler :
			$$X_{n+1}=X_n+\Delta t f(t_{n+1},X_{n+1})$$
			
			\newpage
			
			\item Runge Kutta order 4th : \qquad
			$X_{n+1}=X_n+\frac{\Delta t}{6}\left(K_1+2K_2+2K_3+K_4\right)$ \\
			where \qquad $\left\{\begin{aligned}
				K_1&=f(t^n,X^n) \\
				K_2&=f\left(t^n+\frac{\Delta t}{2},X^n+\frac{1}{2} K_1\Delta t\right) \\
				K_3&=f\left(t^n+\frac{\Delta t}{2},X^n+\frac{1}{2} K_2\Delta t\right) \\
				K_4&=f\left(t^n+\Delta t,X^n+K_3\Delta t\right)
			\end{aligned}\right.$ \\ \; \\ \; \\
			
			\item Scipy function : \qquad \textit{scipy.integrate.solve\_ivp}
		\end{itemize}
		
	\end{frame}
	
	\begin{frame}[allowframebreaks]{Some results (with Python)}
		
		\includegraphics[width=\textwidth]{images/N1000.png} \\ 
		\begin{center}
			\begin{minipage}[c]{0.5\linewidth}
				$\sigma=10,\quad \beta=8/3, \quad r=28$ \\
				$X_0=(-10,10,5)$ 
			\end{minipage}
			$T=100, N = 1000 \Rightarrow \Delta t=0.1$
		\end{center}
		
		\newpage
		
		\includegraphics[width=\textwidth]{images/N100000.png} \\ 
		\begin{center}
			\begin{minipage}[c]{0.5\linewidth}
				$\sigma=10,\quad \beta=8/3, \quad r=28$ \\
				$X_0=(-10,10,5)$ 
			\end{minipage}
			$T=100, N = 100000 \Rightarrow \Delta t=0.001$
		\end{center}
		
		\newpage
		
		\begin{block}{Execution time}
			Until $T=100$ : \\
			\begin{center}
				\begin{tabular}{ll}
					Explicit Euler :&  0.003834 \\
					Implicit Euler :&  0.351176 \\
					Runge Kutta :& 0.228595 \\
					Scipy function :&  0.030235
				\end{tabular}
			\end{center}
		\end{block}
		
	\end{frame}
	
\end{document}

