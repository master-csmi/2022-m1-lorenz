%compile then compile with biber (for the biblio) then recompile
\documentclass[12pt]{article}

\usepackage{a4wide} % increase the typeset area
\usepackage{bm}
\usepackage{amsmath,amssymb}
\usepackage{enumitem}
\usepackage{graphicx}
\usepackage{color}
\usepackage{float} %to place figure H
\usepackage{multirow} %multirow for table
%\usepackage[super]{natbib} %exponant biblio
\usepackage{fourier} %for danger sign
\usepackage{chngcntr}
\usepackage{pifont} %for more symbol in enumerate

% Useful packages for table

\usepackage{array,multirow,makecell}
\usepackage[linesnumbered]{algorithm2e}
\usepackage[table]{xcolor}
\setcellgapes{1pt}
\makegapedcells
\newcolumntype{R}[1]{>{\raggedleft\arraybackslash }b{#1}}
\newcolumntype{L}[1]{>{\raggedright\arraybackslash }b{#1}}
\newcolumntype{C}[1]{>{\centering\arraybackslash }b{#1}}



%hyperref
\usepackage[colorlinks]{hyperref}
\hypersetup{
	colorlinks=true,
	linkcolor=blue,
	filecolor=magenta,      
	urlcolor=blue,
	citecolor=blue
}

% captions
\usepackage{caption}
\newcommand{\vect}[1]{\hat{\boldsymbol{#1}}}
\usepackage{subcaption}
\counterwithin{figure}{section}
\makeatletter
\usepackage[labelformat=simple]{subcaption}
\newcommand\captionsubfigure{%
	\renewcommand\p@subfigure{}
	\renewcommand\thesubfigure{\thefigure.\alph{subfigure}}
}
\makeatother

%to make the appendix
\usepackage{appendix}

%code
\usepackage{listings}
\definecolor{backcolor}{RGB}{240, 240, 240}
\lstdefinestyle{bash}{
	commentstyle=\color{green},
	morecomment=[l][\color{magenta}]{\#},
	backgroundcolor=\color{backcolor},  
	breakatwhitespace=false,
	keepspaces=true,                        
	showspaces=false,                
	showstringspaces=false,
	showtabs=false,                  
	tabsize=1    
}

%for footnote
\usepackage[symbol]{footmisc}
\renewcommand{\thefootnote}{\fnsymbol{footnote}}

%bibliography (with section)
\usepackage[backend=biber,style=numeric,sorting=nyt]{biblatex}
\addbibresource{biblio_ref.bib}
\addbibresource{biblio_doc.bib}


% Titlepage
\newcommand{\reporttitle}{Internship Report : Parareal method and data assimilation for PDEs with Feel++}
\newcommand{\reportauthorOne}{Melissa Aydogdu}
\newcommand{\reportauthorTwo}{Frédérique Lecourtier}
\newcommand{\reportsupervisorOne}{Christophe Prudhomme}
\newcommand{\reportsupervisorTwo}{Luca Berti}
\newcommand{\reporttype}{Coursework}

\begin{document}
	\nocite{*}
	
	\begin{titlepage}

\newcommand{\HRule}{\rule{\linewidth}{0.5mm}} % Defines a new command for the horizontal lines, change thickness here

\begin{center} % Center remainder of the page

\includegraphics[width = 0.5\linewidth]{images/logo-cemosis.pdf}\\[1.5cm] 

\textsc{\Large University of Strasbourg}\\[0.5cm] 
\textsc{\large Master CSMI}\\[0.95cm] 

%----------------------------------------------------------------------------------------
%	TITLE SECTION
%----------------------------------------------------------------------------------------

\HRule \\[0.4cm]
{ \huge \bfseries \reporttitle}\\ % Title of your document
\HRule \\[1.5cm]
\end{center}
%----------------------------------------------------------------------------------------
%	AUTHOR SECTION
%----------------------------------------------------------------------------------------

%\begin{minipage}{0.4\hsize}
\begin{flushleft} \large
	\begin{minipage}{0.4\hsize}
		\textit{Authors:}\\
		\reportauthorOne\\
		\reportauthorTwo
	\end{minipage} \hfill 
	\begin{minipage}{0.4\hsize}
		\textit{Supervisors:}\\
		\reportsupervisorOne\\
		\reportsupervisorTwo
	\end{minipage}
\end{flushleft}
\vspace{4cm}
\makeatletter
Date: \@date 

\vfill % Fill the rest of the page with whitespace



\makeatother


\end{titlepage}


	
	\tableofcontents
	
	\newpage
	\section{Introduction}
	
	\noindent This internship is the continuation of a project realised in th platform Cemosis . During the project the main goals were to implement a parallel-in-time resolution method for the Lorenz system, to realize the data assimilation using the EnKF method using a function already implemented in the library Filterpy. For this we also had to implement several methods to solve numerically the Lorenz system (like RK4). During the project, all these methods were implemented or used in Python.
	
	\subsection{Presentation of Cemosis}
	\label{cemosis}
	
	This project is managed by Cemosis which is the "Centre de Modélisation et de Simulation de Strasbourg" (Strasbourg Modeling and Simulation Center). Cemosis is hosted by the Institute of Advanced Mathematical Research (IRMA) and was created in January 2013. 
	Cemosis work is focused on the numarical simulation and mathematical modeling of different phenomena. They use and develop tools in the fields of:
	
	\begin{enumerate}[label=\textbullet]
		\item \textbf{MSO} - Modeling Simulation and Optimization
		\item \textbf{DS} -	Data Science, Big Data, Smart Data
		\item \textbf{HPC} - High Performance Computing, Parallel Computing, Cloud Computing
		\item \textbf{SI} - Signal and Image processing
	\end{enumerate}
	\noindent They work with researchers and engineers of other research centers and companies.
	
	\noindent For more informations, refer to the \href{http://www.cemosis.fr/}{cemosis website}. 
	\subsection{Context}
	\noindent Cemosis relies currently on the team Modeling and Control of the IRMA and is developing competences and projects in the energy sector of buildings. Nowadays it is important to reduce the energy consumption of buildings in order to move to a more ecologic and economic lifestyle. For this we need to know how to simulate and model buildings using fast simulation methods such as parallel calculation and data integration because our models used to simulate buildings are never perfect, and even less in the case of existing buildings where we have little information. In fact, in order to perform simulations over long periods, such as a year, we would also have to take into consideration phenomena such as radiative exchanges and convective effects, which therefore require a PDE model if we want a spatial discretization.
	\begin{figure}[H]       
	\begin{minipage}[t]{0.48\linewidth}
		\centering
		\includegraphics[width=\linewidth]{"images/cemosis_simulation_1.png"}
	\end{minipage} \hfill
	\begin{minipage}[t]{0.48\linewidth}
		\centering
		\includegraphics[width=\linewidth]{"images/cemosis_simulation_2.png"}
	\end{minipage}
	\captionof{figure}{Example of a building simulation with Feel++}
    \end{figure}
	
	\subsection{Goals of the Intership}
	\noindent The main objective of the intership is to implement the Parareal method and data assimilation for PDEs with Feel++. 
	\newline
	
	\noindent \textbf{Objectives for the common part}:
    	\begin{enumerate}
        \item To read the following article about the Heat equation (Chapter 11) : \cite{quarteroni_numerical}
			\item To set up a project on Github :
			\begin{itemize}
				\item Organisation of the repository (Python library, cmake in C++ ...)
				\item Create issues to see the progress of the project
				\item Set up the CI : build, test, documentation 
			    \end{itemize}\; \\
		Github repository : \url{https://github.com/master-csmi/2022-m1-lorenz}
	\end{enumerate}
	
	\noindent \textbf{Objectives for parareal method:}
	\begin{enumerate}
		\item Implement the parareal method in C++ and :
		\begin{itemize}
			\item Test the method (with oscillator)
			\item Check convergence and stability results
			\item Check speed-up and efficiency 
		\end{itemize}
		\item Implement the resolution of the heat equation in C++ with Feel++ \\
		$\Rightarrow \quad $ Implement the resolution of the Laplace equation in C++ with Feel++
		\item Use the previous implementation of the heat equation with the parareal method
		\item Check the convergences/accuracies of the method
	\end{enumerate}
	\newpage
	\noindent \textbf{Objectives for the Data assimilation:}
	\begin{enumerate}
        \item Write a class for the EnKF in C++, inspired by the EnKF written in Python (FilterPy); Test the algorithm.
        \item To read the following article :
        \begin{itemize}
            \item Fundamentals Of Building Performance Simulation By Beausoleil-Morrison
        \end{itemize}
        
    \item Understand the heat equation and what are the phenomena involved in the modification of the temperature of a building (conduction, convection, radiation).
    \item Write the mathematical problem to be solved if we want to simulate an office then realize the simulation using Feel++ toolboxes.
    \item Introduce data assimilation using the sensor and correct the simulation.
    \end{enumerate}

	\newpage

	\section{Differential Equations}
	\noindent In Mathematics, ordinary differential equations (ODE) are equations that involve derivatives of one-variable functions, and partial differential equations (PDE) are equations that imposes relations between the various partial derivatives of a multivariable function.
	The difference between ODEs and PDEs is that for ODEs the unknown functions depend only on one variable, whereas for PDEs the unknown functions may depend on several independent variables.
	\noindent Differential equations are an important object of study in both pure and applied mathematics. They are used to build mathematical models of physical and biological evolution processes, for example for the study of radioactivity, celestial mechanics, weather or population dynamics... 
	\noindent During the project we had already used the Lorenz system and the harmonic oscillator, for the internship we also used these two differential equations but we worked with two other equations: the heat equation and the Laplacian.
	
	\subsection{ODE: Harmonic oscillator}
	\label{oscillator_ode}
	\noindent A harmonic oscillator is an ideal oscillator whose evolution over time is described by a sinusoidal function, whose frequency depends only on the characteristics of the system and whose amplitude is constant.This mathematical model describes the evolution of any physical system in the vicinity of a stable equilibrium position, which makes it a transversal tool used in many fields: mechanics, electricity and electronics, optics.
	\begin{figure}[H]       
	\begin{minipage}[t]{0.46\linewidth}
		\centering
		\includegraphics[width=\linewidth]{"images/Diff_equation/Harmonic_oss_1.png"}
		\captionof{figure}{Simple pendulum}
	\end{minipage} \hfill
	\begin{minipage}[t]{0.48\linewidth}
		\centering
		\includegraphics[width=\linewidth]{"images/Diff_equation/Harmonic_oss_2.png"}
		\captionof{figure}{Spring/mass system}
	\end{minipage}
    \end{figure}
	
	\newpage
	
	\noindent Let's consider a mass-spring system of the following form :

    \begin{equation}
    	\frac{d^2 x}{d t^2}+\omega_0^2 x = 0 \quad \iff \quad \frac{d^2 x}{d t^2}=-\omega_0^2 x \quad \text{with} \quad \omega_0=\sqrt{\frac{k}{m}}.
    	\label{osc}
    \end{equation}
    
    \noindent $k$ and $m$ are the spring stiffness and the suspended mass respectively. We are interested in this equation because its exact solution is known and is of the form :
    $$x(t) = x_0 \cos(\omega_{0}t+\phi_0).$$
    
    \noindent First of all, the numerical methods such as Runge Kutta order 4  allow us to solve first order differential equations. But the harmonic oscillator equation (\ref{osc}) is a second order equation. We will therefore start by making a simple change of variable which will allow us to replace this second order differential equation by a system of two first order equations. We pose :
     
    $$\qquad \frac{d x}{d t}=v \quad \Rightarrow \quad \frac{d^2 x}{d t^2}=\frac{d v}{d t}.$$
    
    \noindent As a result the equation becomes :
    
    $$\left\{\;\begin{aligned}
    	\frac{d x}{d t}&=v \\
    	\frac{d v}{d t}&=-\omega_0^2 x
    \end{aligned}\right.
    $$
    
    \noindent Let's take an example to understand how we can determine the exact solution from the initial conditions that we will take. For example if we take $\omega_0=5$ and $(x(0),v(0))=(0,1)$, we have :
    
    $$\left\{\;\begin{aligned}
    	x(0)&=0 \\
    	v(0)&=1
    \end{aligned}\right. \quad \Rightarrow 
    \left\{\;\begin{aligned}
    	x_0 \cos(\phi_0)&=0 \\
    	-x_0 \omega_{0} \sin(\phi_0)&=1
    \end{aligned}\right.  \quad \Rightarrow  
    \left\{\;\begin{aligned}
    	x_0&=\frac{-1}{5} \\
    	\phi_0&=\frac{\pi}{2}
    \end{aligned}\right.
    $$
    
    \noindent And thus the solutions of the equation are of the form :
    
    $$x(t) = \frac{-1}{5}\cos(5t+\frac{\pi}{2}).$$
    \subsection{ODE: Lorenz system}
	\label{lorenz_ode}
	\noindent The Lorenz system is a simplified three-variable model to investigate atmospheric convection. This model has had important repercussions in showing the possible limits on the ability to predict long-term climate and weather evolution. One of the important characteristics of the Lorenz system is 
	that it is a chaotic system, which means that this type of system is roughly defined by sensitivity to initial conditions: infinitesimal differences in initial conditions of the system result in large differences in behavior.
	\begin{figure}[H]   
		\centering
		\includegraphics[width=0.5\textwidth]{"images/butterfly.jpg"}
		\caption{butterfly wing pattern}
		\label{but_wing}
	\end{figure}	
	
	\noindent The Lorenz system defines a 3 dimensional trajectory by differential equations with 3 parameters.
	$$
	\begin{cases}
		
		x'&=\sigma(y-x) \\
		y'&=x(r-z)-y \\
		z'&=xy-bz
		
	\end{cases}
	$$
	
	\noindent Here, $x$ is proportional to the rate of convection, $y$ is related to the horizontal temperature variation, and $z$ is the vertical temperature variation.
	We have also three parameters all strictly positive:
	\begin{enumerate}[label=\textbullet]
		\item $\sigma > 0$  relates to the Prandtl number. This number is a dimensionless quantity that puts the viscosity of a fluid in correlation with the thermal conductivity;
		\item $r > 0$  relates to the Rayleigh number, it is a control parameter, representing the temperature difference between the top and bottom of the tank;
		\item $b > 0$ relates to the physical dimensions of the layer of fluid uniformly heated from below and cooled from above.
	\end{enumerate}
	
	\noindent We can see that this system is non-linear, because in the second differential equation ( $\frac{dy}{dt}$) we can see the term $xz$ and in the third differential equation ($\frac{dz}{dt}$) we have $xy$. The three differential equations form a coupled system. 
	
	\noindent Let us now determine the fixed points of the Lorenz system. These are the points such that $X'=0$. 
	
	$$
	X'=(x',y',z')=0
	\quad \Rightarrow \quad 
	\left\{\begin{aligned}
		x'=&\sigma(y-x) &&=0 \\
		y'=&x(r-z)-y  &&=0\\
		z'=&xy-bz &&=0
	\end{aligned}\right. 
	\quad \Rightarrow \quad 
	\left\{\begin{aligned}
		&x=y \\
		&(r-1-z)x=0\\
		&x^2=bz
	\end{aligned}\right.
	$$
	
	\begin{enumerate}[label=\textbullet]
		\item If $x=0$ : \quad  $y=0$ and $z=0$.
		\item If $x\ne 0$ and $r>1$ : \quad $\left\{\begin{aligned}
			&z=r-1\\
			&x=y=\pm\sqrt{b(r-1)}
		\end{aligned}\right.
		$
	\end{enumerate}
	
	\noindent We deduce that the fixed points of the Lorenz system are: $(0,0,0)$ for all values of the parameters. And for $r>1$, there is also a pair of fixed points $(\sqrt{b(r-1)},\sqrt{b(r-1)},r-1)$ and $(-\sqrt{b(r-1)},-\sqrt{b(r-1)},r-1)$.
	
	\subsection{PDE: Laplacian equation}
	\label{laplacian_pde}
    \noindent The Laplace equation is a second-order partial differential equation . This equation is a basic PDE that arises in the heat and diffusion equations. It is a useful method for determining electric potentials in space or the free region.
    It is often written as :
    $$\Delta f=0 \qquad \text{or} \qquad \nabla^2 f=0$$
    \noindent with $\Delta$ the Laplace operator. We can define the Laplace operator as follows: $\Delta=\nabla \cdot \nabla$ where $\nabla \cdot$ divergence operator and $\nabla$  is the gradient operator.
    
    \noindent We can write the problem this way:
    
	$$\left\{\begin{aligned}
		-\Delta u &= f \quad&&\Omega \\
		u&=g \quad&&\Gamma_D \\
		\frac{\partial u}{\partial n} &=h \quad &&\Gamma_N \\
		\frac{\partial u}{\partial n}+u &=l \quad &&\Gamma_R \\
	\end{aligned}\right.$$
	\noindent Where $\Omega$ corresponds to our domain, $\Gamma_D$  corresponds to the Dirichlet boundary condition, $\Gamma_N$ to the Neumann boundary condition and $\Gamma_R$ to the Robin boundary condition.
	\subsection{PDE: Heat equation}
	\label{heat_pde}
	\noindent Heat transfer is the process of energy transfer resulting from a temperature difference. 
	\newline
    \noindent Thermal analysis is undertaken to predict temperatures and heat transfer in and around bodies. This information can then be used to model temperature-dependent phenomena, such as heat-induced stresses or the effect on fluid flow in the case of a solidifying metal.  Heat flow has been classified into three different modes: conduction, convection and radiation.
    \begin{figure}[H]       
	\begin{minipage}[t]{0.45\linewidth}
		\centering
		\includegraphics[width=\linewidth]{"images/Diff_equation/heat_1.png"}
	\end{minipage} \hfill
	\begin{minipage}[t]{0.50\linewidth}
		\centering
		\includegraphics[width=\linewidth]{"images/Diff_equation/heat_2.png"}
	\end{minipage}
	\captionof{figure}{Simulation of the temperature of a building using the heat equation with Feel++}
    \end{figure}


	\noindent The heat equation with convective effects can be written as:
    $$\rho C_p((\frac{\partial T}{\partial t})+u . \nabla T)-\nabla .(k \nabla T)=Q$$
    and it must be completed  with boundary conditions and initial conditions.
    \newline
	\renewcommand{\arraystretch}{2}
    \begin{tabular}{|R{3cm}|C{3cm}|L{3cm}|L{3cm}|}
    \hline
    Notation & Quantity &Unit & Value   \\
    \hline
    $\rho$ & density & $Kg.m^
    {-3}$ & 1.125  \\[4cm]
    \hline
    $C_p$ & Specific heat & $J/KgC=J/KgK$ & 1004 \\[4cm]
    \hline
    $k$ & Conductivity & $W/mC=W/mK$ & 0.025 \\[4cm]
    \hline
    $u$ & Fluid velocity & $m.s^{-1}$  & unknown \\[4cm]
    \hline
    $T$ & Temperature & $K$ or $C$  & unknown \\[4cm]
    \hline
    $t$ & Time & s. &   \\[4cm]
    \hline
\end{tabular}

	\newpage

    \subsection{Runge-Kutta}
    \label{rk4}
    \noindent During our project and internship we used the Runge-Kutta method of order 4 to solve ordinary differential equations. Runge-Kutta techniques are one-step numerical schemes for solving ordinary differential equations. They are among the most popular methods because of their ease of implementation and accuracy.
    
    \noindent We consider $f : [0; T] \times \mathbb{R}^n \rightarrow \mathbb{R}^n$ a continuous function. For $X_0\in \mathbb{R}^n$, the problem is to find  $X\in C^1([0,T],\mathbb{R}^n)$ solution for the differential equation:
	
	$$\left\{\begin{aligned}
		X'&=f(t,X), \\
		X(0)&=X_0.
	\end{aligned}\right.$$
	
	
	 \noindent After discretizing the problem in time, we can use the Runge Kutta method of order 4 to solve the ODE:
		
		$$X_{n+1}=X_n+\frac{\Delta t}{6}\left(K_1+2K_2+2K_3+K_4\right) ,$$
		
		\noindent where 
		
		$$\left\{\begin{aligned}
			K_1&=f(t_n,X_n) , \\
			K_2&=f\left(t_n+\frac{\Delta t}{2},X_n+\frac{1}{2} K_1\Delta t\right) , \\
			K_3&=f\left(t_n+\frac{\Delta t}{2},X_n+\frac{1}{2} K_2\Delta t\right) , \\
			K_4&=f\left(t_n+\Delta t,X_n+K_3\Delta t\right) .
		\end{aligned}\right.$$
	
    

	%\noindent For the next part of the project, we choose to work with the Runge kutta method.
	
	\newpage
	
	\section{Para-real method}
	
Para-real method is a parallel-in-time integration methods which was introduced in 2001 by Lions, Maday and Turinici. Parareal computes the numerical solution for multiple time steps in parallel, it is categorized as a parallel across the steps method.

\subsection{Explanation}

As for the previous methods, we consider an initial value problem of the form \\
\begin{minipage}{\linewidth}
	\centering
	$\left\{\begin{aligned}
		X'&=f(t,X) \qquad t_0\le t\le T \\
		X(t_0)&=X_0
	\end{aligned}\right.$ \\
\end{minipage} \\

\noindent Parareal method need a decomposition of the time interval $[t_0,T]$ into $P$ slices $[t_j,t_{j+1}]$ with  $j\in\{0,\dots,P-1\}$ where $P$ is the number of process units. We want to parallelize the algorithm so each time slice is assigned to one process. We denote by $F$ a function which is of high accuracy and $G$ which is of low accuracy. Also, $F$ will be very expensive in terms of calculation but very accurate and $G$ will be very cheap but very imprecise. We denote by $\Delta t_F$ the fine time step and by $\Delta t_G$ th coarse time step. To have the right number of points in total (i.e. the sum of the number of points of each interval is equal to the number of points between $t_0$ and $T$), we are not going to cut the interval in equal parts but the $t_j$ will have to be multiples of $\Delta t_G$ (we will choose the multiple of $\Delta t_G$ closest to $t$ which cuts our interval in equal parts).
For example, taking $P=3$ processes, $\Delta t_G=0.1$ and an interval from $t_0=0$ to $T=2$, our exact $t_j$ will be : $[0,\;0.666,\;1.333,\;2]$ which we will be approximated  by $[0,\;0.7,\;1.3,\;2]$. We will also suppose that the coarse time step $\Delta t_G$ is a multiple of the fine time step $\Delta t_F$. If this was not the case, a simple way to avoid the problems related to the number of points could be an interpolation between the two time steps $\Delta t_F$ closest to $\Delta t_G$. \\

\noindent We denote by $U_j^k$, $j\in\{0,\dots,P\}$ the initial point at time $t_j$ and at iteration k. We also note $F(U_{j-1}^k)$, $j\in\{1,\dots,P\}$ the fine integrator between $t_{j-1}$ and $t_j$ which start by the initial point $U_{j-1}^k$ at iteration k and respectively $G(U_{j-1}^k)$, $j\in\{1,\dots,P\}$ the coarse integrator between $t_{j-1}$ and $t_j$ which start by the initial point $U_{j-1}^k$ at iteration k. Then, a series of steps (\textit{see \ref{parareal}}) is performed until the solution of the system converges. \\

\noindent At iteration $k=0$ :
\begin{enumerate}[label=\textbullet]	
	\item Step 1 (\textit{see \ref{parareal:1}}) : At iteration $k=0$, we have an initial point $U_0^0=X_0$.
	\item Step 2 (\textit{see \ref{parareal:2}}) : We start by applying the function $G$ on all intervals $[t_j,t_{j+1}]$ and we note $U_j^0=G(U_{j-1}^0)$ the values of $G$ at $t_j$. \\
	Note that this part of the method can be done sequentially because if we parallelize the task, the process $j$ should wait until the process $j-1$ has finished before starting.
	\item Step 3 (\textit{see \ref{parareal:3}}) : We can then calculate from each $U_j^0$ the fine solution between $t_j$ and $t_{j+1}$ : $F(U_j^0)$. This is an operation that must be parallelized.
\end{enumerate}

\noindent At iteration $k=1$ :
\begin{enumerate}[label=\textbullet]	
	\item Step 4 (\textit{see \ref{parareal:4}}) : We can then continue to iteration $k=1$ where we need the values $G(U_j^0)$ and $F(U_j^0)$ calculated at the previous iteration ($k=0$). \\
	We will also keep the initial point at time $t_0$ : $U_0^1=U_0^0$.
	\item Step 5 (\textit{see \ref{parareal:5}}) : We can then calculate $G(U_0^1)$ which allows us to obtain the point $U_1^1$ by the following formula:
	$$U_j^1=G(U_{j-1}^1)+(F(U_{j-1}^0)-G(U_{j-1}^0))$$
	Note that due to $U_0^1=U_0^0$, we have $G(U_0^1)=G(U_0^0)$ and therefore $U_1^1=F(U_0^0)$ \\
	We then compute in the same way the following $G(U_j^1)$ and the associated $U_{j+1}^1$ points. This step can be done sequentially for the same reason as in step 2.
	\item Step 6 (\textit{see \ref{parareal:6}}) : We can then calculate from each $U_j^1$ the fine solution between $t_j$ and $t_{j+1}$ : $F(U_j^1)$. This is an operation that must be parallelized. \\
	Note that due to $U_0^1=U_0^0$, we also have $F(U_0^1)=F(U_0^0)$.
\end{enumerate}

\noindent Then we repeat steps 3 to 6 until $U_j^k-U_j^{k-1}\rightarrow 0 \quad \forall j\in\{0,\dots,P-1\}$. We have :
$$U_j^k=G(U_{j-1}^k)+(F(U_{j-1}^{k-1})-G(U_{j-1}^{k-1}))$$

\begin{figure}[h]
	\captionsubfigure
	\fbox{
		\centering \qquad
	\begin{minipage}[c]{\linewidth}
	\begin{subfigure}[t]{.30\linewidth}       
		\includegraphics[width=\linewidth]{"images/parareal/parareal_1.jpg"}
		\captionof{figure}{ : Step 1}
		\label{parareal:1}
	\end{subfigure} 
	\begin{subfigure}[t]{.30\linewidth}       
		\includegraphics[width=\linewidth]{"images/parareal/parareal_2.jpg"}
		\captionof{figure}{ : Step 2}
		\label{parareal:2}
	\end{subfigure} 
	\begin{subfigure}[t]{.30\linewidth}       
		\includegraphics[width=\linewidth]{"images/parareal/parareal_3.jpg"}
		\captionof{figure}{ : Step 3}
		\label{parareal:3}
	\end{subfigure} 
	
	\begin{subfigure}[t]{.30\linewidth}       
		\includegraphics[width=\linewidth]{"images/parareal/parareal_4.jpg"}
		\captionof{figure}{ : Step 4}
		\label{parareal:4}
	\end{subfigure} 
	\begin{subfigure}[t]{.30\linewidth}  
		\includegraphics[width=\linewidth]{"images/parareal/parareal_5.jpg"}
		\captionof{figure}{ : Step 5}
		\label{parareal:5}
	\end{subfigure} 
	\begin{subfigure}[t]{.30\linewidth}      
		\includegraphics[width=\linewidth]{"images/parareal/parareal_6.jpg"}
		\captionof{figure}{ : Step 6}
		\label{parareal:6}
	\end{subfigure}
	\end{minipage}}
	\caption{Parareal method}
	\label{parareal}
\end{figure}

\newpage

\noindent \underline{\textit{Notes :}} Following k iteration.
\begin{itemize}
	\item We have : $\qquad U_0^k=U_0^0=X_0 \quad \forall k$.
	\item Note that the first 2 steps are always done because there can't be convergence with only one value. So for example if we take only one process and we apply the parareal method, we will have at the first iteration ($k=0$) only one initial point $U_0^0=X_0$ and we will compute the fine and coarse solution only on this point. We can then go to the next iteration which will be exactly the same and the algorithm will stop immediately. Indeed, there will be convergence between $U_0^0$ and $U_0^1$ (because they are equal), moreover there is obviously convergence between the 2 solutions. However, there is still one extra iteration ($k=1$) and the computation of the coarse solution is also useless because it is not used to compute the other initial points due to the fact that there is only one.
	\item We can also notice that the calculations which are done on the last interval $[t_{P-1},t_P]$ are not used because the points $U_P^k$ are not useful for the method. On this interval, one could then compute the fine solution only if there is convergence.
\end{itemize}

\subsection{Application to the harmonic oscillator}

\noindent Before moving on to the Lorenz system, we will consider a mass-spring system of the following form :

\begin{equation}
	\frac{\partial^2 x}{\partial t^2}+\omega_0^2 x = 0 \quad \iff \quad \frac{\partial^2 x}{\partial t^2}=-\omega_0^2 x \quad \text{with} \quad \omega_0=\sqrt{\frac{k}{m}}
	\label{osc}
\end{equation}

\noindent $\omega_0$ is called the natural pulsation of the harmonic oscillator. $k$ and $m$ are the spring stiffness and the suspended mass respectively. We are interested in this equation because its exact solution is known and is of the form :

$$x(t) = x_0 \cos(\omega_{0}t+\phi_0)$$

\noindent First of all, the numerical methods we have seen in Section \ref{sec4} (such as Runge Kutta order 4 in section \ref{sec4.4}), allow us to solve first order differential equations. But the harmonic oscillator equation \ref{osc} is a second order equation. We will therefore start by making a simple change of variable which will allow us to replace this second order differential equation by a system of two first order equations. We pose :
 
$$\qquad \frac{\partial x}{\partial t}=v \quad \Rightarrow \quad \frac{\partial^2 x}{\partial t^2}=\frac{\partial v}{\partial t}$$

\noindent And so the equation becomes :

$$\left\{\;\begin{aligned}
	\frac{\partial x}{\partial t}&=v \\
	\frac{\partial v}{\partial t}&=-\omega_0^2 x
\end{aligned}\right.
$$

\noindent Let's take an example to understand how we can determine the exact solution from the initial conditions that we will take. For example if we take $\omega_0=5$ and $(x(0),v(0))=(0,1)$, we have :

$$\left\{\;\begin{aligned}
	x(0)&=0 \\
	v(0)&=1
\end{aligned}\right. \quad \Rightarrow 
\left\{\;\begin{aligned}
	x_0 \cos(\phi_0)&=0 \\
	-x_0 \omega_{0} \sin(\phi_0)&=1
\end{aligned}\right.  \quad \Rightarrow  
\left\{\;\begin{aligned}
	x_0&=\frac{-1}{5} \\
	\phi_0&=\frac{\pi}{2}
\end{aligned}\right.
$$

\noindent And thus the solutions of the equation are of the form :

$$x(t) = \frac{-1}{5}\cos(5t+\frac{\pi}{2})$$

\noindent We want now to apply the parareal method on the harmonic oscillator. For the two integrators we will take Runge Kutta order 4 method with a little time step for $F$ and a larger one for $G$. \\

\colorbox{yellow}{TO COMPLETE : convergence of the method}

\subsection{Application to the Lorenz system}

We try now to apply this method on the Lorenz system. In the previous explanation, we have also :

$$X'=\begin{pmatrix}
    x' \\
    y' \\
    z'
\end{pmatrix}, \quad X=\begin{pmatrix}
    x \\
    y \\
    z
\end{pmatrix} \quad et \quad f(t,X)=\begin{pmatrix}
    \sigma(y-x) \\
    x(r-z)-y \\
    xy-bz
\end{pmatrix}$$

\noindent For the two integrators we will take Runge Kutta order 4 method with a little time step for $F$ and a larger one for $G$. \\

\colorbox{yellow}{TO COMPLETE : example of result}

	
	\newpage
	
	

\begin{frame}[allowframebreaks]{Objectives}
		
    \begin{enumerate}[\textbullet]
        \item Write a class for the EnKF in C++, inspired by the EnKF written in Python (FilterPy); Test the algorithm.
        \item To read the following article :
        \begin{itemize}
            \item Fundamentals Of Building Performance Simulation By Beausoleil-Morrison
        \end{itemize}
        
    \item Understand the heat equation and what are the phenomena involved in the modification of the temperature of a building (conduction, convection, radiation).
    \item Write the mathematical problem to be solved if we want to simulate an office then realize the simulation using Feel++ toolboxes.
    \item Introduce data assimilation using the sensor and correct the simulation.
    \end{enumerate}
\end{frame}




\begin{frame}{Introduction to data assimilation}
Data assimilation is widely used in:
\begin{enumerate}[\textbullet]
       \item weather forecasting
       \item ocean simulation
\end{enumerate}	 
      The main idea of computational data assimilation is to combine:
\begin{enumerate}[\textbullet]
       \item a model
       \item some observations
\end{enumerate}	 
The best estimation:
$$x^a=Lx^b+Ky^0$$
with $x^a$ the analyzed state, $x^b$ the state of the model and $y^0$ the observations.
\end{frame}
\subsection{Theory}
   
\begin{frame}{Kalman filter in multi-dimensional case}
   Now that we have explained the method for finding $x^a$ let's try to generalize our formula in a \textbf{multi-dimensional case}.

   $$\left\{\begin{aligned}
     &x^a=(I-KH)x^b+Ky^0=x^b+K(y^0-H(x^b)) \\
           &K=BH^T(HBH^T+R)^{-1} \\
    \end{aligned}\right.$$
   With $K$ the gain or weight matrix.
   \begin{figure}[H]
       \pgfimage[width=0.4\linewidth]{images/enkf/schema_kalman_filter.png}
       \caption{Kalman Filter}
   \end{figure}
\end{frame}
\begin{frame}[allowframebreaks]{Ensemble Kalman Filter}
   \noindent The ENKF method consists in using the Kalman filter method in high dimension and compare P by a set of states $x_1,x_2,..,x_{m}$. So we can approximate the moments of the error by the moments of the sample.
For all the samples, we have:
$$x_i^a=x_i^f+K[y-h(x_i^f)]$$
with $h(x_i^f)$ the observation operator.
\newline \noindent To begin with we can estimate the
forecast error covariance matrix as:
$$P^f=\frac{1}{m-1}\sum_{i=1}^{m}(x_i^f-\bar{x}^f)(x_i^f-\bar{x}^f)^T~~with~~\bar{x}^f=\frac{1}{m}\sum_{i=1}^{m}x_i^f .$$ 
\noindent We can factorized the forecast error covariance matrix by:
$$P^f=X_f X_f^T$$
where $X_f$ is an $n \times m$ matrix whose columns are the normalized anomalies or normalized perturbations,
$$[X_f]_i=\frac{x_i^f-\bar{x}^f}{\sqrt{m-1}}$$

\noindent We can also define the Kalman gains: 
$$K=P^f H^T(HP^f H^T+R)^{-1}$$

\noindent In addition, we have:
$$
\bar{x}^a=\frac{1}{m}\sum_{i=1}^mx_i^a~~,~~~~[X_a]_i=\frac{x_i^a-\bar{x}^a}{\sqrt{m-1}}. $$
\newline \noindent We can write the covariance matrix of the analysis errors as:
$$P^a=(I_n-KH)P^f(I_n-KH)^T+KRK^T=(In-KH)P^f.$$
\noindent Add a disruption to the observation: $y\rightarrow y_i+\bar{u_i}$ where $u_i$ is drawn from the Gaussian distribution $u_i \sim \mathcal{N}(0,R)$
\newline the innovation perturbations as :
$$[Y_f]_i=\frac{Hx_i^f-u_i-H\bar{x}^f+\bar{u}}{\sqrt{m-1}}.$$
\noindent Finally we can modify the posterior anomalies:
$$X_i^a=X_i^f-KY_i^f=(I_n-KH)X_i^f+\frac{K(u_i-\bar{u})}{\sqrt{m-1}}.$$
\end{frame}
\begin{frame}{Ensemble Kalman Filter}
    \begin{figure}[H]
		\pgfimage[width=0.85\linewidth]{images/enkf/algorithm.png}
	\end{figure}
\end{frame}
\begin{frame}[allowframebreaks]{Comparing Cpp results with Python}
\begin{minipage}{0.4\hsize}
			\centering
	\begin{figure}[H]
		\pgfimage[width=0.9\linewidth]{images/enkf/result.png}
		\caption{Visualization of the model and observations (Initial condition: $X_0=(-10.,-10.,25.) $ )}
	\end{figure}
		\end{minipage} \quad
		\begin{minipage}{0.5\hsize}
			\begin{itemize}
    \item $(\sigma, r, b)=(12.,6.,12.)$ for observation
    \item $(\sigma, r, b)=(10.,6.,10.)$ for the model
    \item $$P=\begin{pmatrix}
            0.1 & 0. & 0. \\
            0. & 0.1 & 0. \\
            0. & 0. & 0.1 \\
            \end{pmatrix}
            Q=\begin{pmatrix}
            0.1 & 0. & 0. \\
            0. & 0.1 & 0. \\
            0. & 0. & 0.1 \\
            \end{pmatrix}$$
            \newline
            $$R=\begin{pmatrix}
            0.01 & 0. & 0. \\
            0. & 0.01 & 0. \\
            0. & 0. & 0.01 \\
            \end{pmatrix}.$$ 
    \end{itemize}
		\end{minipage}
		
\newpage
\centering
	\begin{figure}[H]
		\pgfimage[width=1\linewidth]{images/enkf/result_filterpy.png}
		\caption{Result with Filterpy}
	\end{figure}

\newpage
\centering
	\begin{figure}[H]
		\pgfimage[width=1\linewidth]{images/enkf/result_cpp.png}
		\caption{Result with Cpp}
	\end{figure}

\end{frame}
\subsection{Integrate data assimilation to Feel++}
\begin{frame}[allowframebreaks]{The context}
\begin{itemize}
    \item Our goal is to make the thermal simulation of an office in the university of Strasbourg in which we have 10 sensors to measure the temperature.\\
    \item We want to apply data assimilation and use our sensors to correct the temperature of the room.
\end{itemize}

\begin{minipage}{0.48\linewidth}
    \begin{figure}
        \centering
        \includegraphics[width=\linewidth]{"images/enkf/Maillage_1.png"}
        \caption{Mesh of the office that we will use for the simulation.}
    \end{figure}
\end{minipage} \;
\begin{minipage}{0.48\linewidth}
    \begin{figure}
        \centering
        \includegraphics[width=\linewidth]{"images/enkf/Maillage_2.png"}
        \caption{Visualization of windows and door in the mesh.}
    \end{figure}
\end{minipage}
\newpage
\begin{itemize}
    \item Heat equation with convective effects\\
    $$\rho C_p((\frac{\partial T}{\partial t})+u \cdot \nabla T)-\nabla \cdot (k \nabla T)=Q$$
    Which is completed with boundary conditions and initial value.
    \newline
    \newline
\renewcommand{\arraystretch}{2}
\begin{tabular}{|R{2cm}|C{2.5cm}|L{2.5cm}|L{2.5cm}|}
\hline
$\rho$ & Air density & $Kg.m^
{-3}$ & 1.125  \\[0.5cm]
\hline
$C_p$ & Specific heat & $J/KgK$ & 1004 \\[0.5cm]
\hline
$k$ & Conductivity & $W/mK$ & 0.025  \\[0.5cm]
\hline
$u$ & Fluid velocity & $m.s^{-1}$ & unknown \\[0.5cm]
\hline
\end{tabular}
\newpage
\item Equation of the air motion (Navier-Stokes).
 $$\left\{\begin{aligned} 
        &\rho (\frac{\partial u}{\partial t}+u\cdot \nabla u)-\nabla \cdot (\mu \nabla u)+\nabla P =-\rho_0 \beta(T-T_{ref})g\\
        &\nabla . u=0 \\
    \end{aligned}\right.$$
\renewcommand{\arraystretch}{2}
\begin{tabular}{|R{1cm}|C{5cm}|L{1.5cm}|L{1.5cm}|}
\hline
$\rho$ & fluid density & $Kg.m^
{-3}$ & 1.125 \\[0.7cm]
\hline
$u$ & fluid velocity & $m.s^{-1}$ & unknown \\[0.7cm]
\hline
$\beta$ & coefficient of thermal expansion & $K^
{-1}$ &  0.0034 \\[0.7cm]
\hline
$\mu$ & dynamic viscosity & $Pa.s$ & $1.81e^{-5}$\\[0.7cm]
\hline
$g$ & gravitational acceleration & $m.s^{-2}$ & $9.8$\\[0.7cm]
\hline
\end{tabular}

\newpage
\item Heat equation without convective effects\\
    $$\rho C_p(\frac{\partial T}{\partial t})-\nabla \cdot (k \nabla T)=Q$$
    Which is completed with boundary conditions and initial value.
    \newline
    \newline
\renewcommand{\arraystretch}{2}
\begin{tabular}{|R{2cm}|C{2.5cm}|L{2.5cm}|L{2.5cm}|}
\hline
$\rho$ & Air density & $Kg.m^
{-3}$ & 1.125  \\[0.5cm]
\hline
$C_p$ & Specific heat & $J/KgK$ & 1004 \\[0.5cm]
\hline
$k$ & Conductivity & $W/mK$ & 0.025  \\[0.5cm]
\hline
\end{tabular}
\end{itemize}
\newpage
\begin{minipage}{0.30\linewidth}
    \begin{figure}
        \centering
        \includegraphics[width=1.2\linewidth]{images/enkf/conduction.png}
        \caption{Conduction}
    \end{figure}
\end{minipage} \;
\begin{minipage}{0.30\linewidth}
    \begin{figure}
        \centering
        \includegraphics[width=1.2\linewidth]{images/enkf/convection.png}
        \caption{Convection}
    \end{figure}
\end{minipage}
\begin{minipage}{0.35\linewidth}
    \begin{figure}
        \centering
        \includegraphics[width=1.2\linewidth]{images/enkf/radiation.png}
        \caption{Radiation}
    \end{figure}
\end{minipage}

\end{frame}
\begin{frame}[allowframebreaks]{Simulation}
\begin{minipage}{0.47\linewidth}
    \begin{figure}
        \centering
        \includegraphics[width=\linewidth]{"images/enkf/sim_1.png"}
        \caption{Visualization of the simulation made with the toolbox heat (Initial condition ) .}
    \end{figure}
\end{minipage} \;
\begin{minipage}{0.48\linewidth}
    \begin{figure}
        \centering
        \includegraphics[width=\linewidth]{"images/enkf/sim_2.png"}
        \caption{Visualization of the simulation made with the toolbox heat (1 hour) .}
    \end{figure}
\end{minipage}
\begin{itemize}
    \item For this simulation we do not consider the convection .
\end{itemize}

\end{frame}
    
\begin{frame}[allowframebreaks]{Including data assimilation to our simulation}
\begin{minipage}{1\linewidth}
    \begin{figure}
        \centering
        \includegraphics[width=\linewidth]{"images/enkf/enkf.png"}
    \end{figure}
\end{minipage}
Ensemble Kalman Filter class:
\begin{itemize}
    \item the dimension of our model;
    \item the dimension of the vector containing our observations;
    \item a vector X which will represent the initial state for our model;
    \item the matrix P ;
    \item our time step dt. 
    \item a function \text{hx};
    \item a function \text{fx}.

\end{itemize}
\newpage
The delicate part to realize the data assimilation was to create the fx function.

	\begin{figure}[H]
		\pgfimage[width=0.6\linewidth]{images/enkf/Maillage_rayon.png}
		\caption{Mesh with radius on the sensors}
	\end{figure}


\end{frame}




	

	\newpage

	\section{Conclusion}
	\noindent Our goal during the project was to implement a parallel time resolution method for the Lorenz system, and to realize the data assimilation using the EnKF method.
	For the internship which is a continuation of the project, we had to implement these two methods in C++ in order to integrate and test them with Feel++.
	\noindent We had to implement the parareel method and the data assimilation in order to test them with ODEs like the harmonic oscillator or the Lorenz system but also with PDEs like the heat equation or the Laplacian. Since the implementation was already done in Python during the project, we could check the results obtained with C++ and compare them with those obtained in Python.
	
	\noindent For the parareel part...
	
	\noindent For the data assimilation part, we first had to realize the implementation of the Ensemble Kalman Filter algorithm in C++ based on the one already done in Python in the FilterPy library. We were able to compare the results by applying it to the Lorenz system to verify our algorithm. After making sure that our EnKF in C++ was working properly, we had to understand the heat equation in order to perform a thermal simulation of an office located at the UFR. Furthermore, we had data from the sensors in the office at our disposition. Using the simulation performed with Feel++ as the model and the data as the observations, we had to perform data assimilation to correct the model. We noticed that there were some incorrect values during this part, but due to lack of time we did not manage to correct our algorithm. For the future, it would be interesting to look further into the problem we had during the realization of the data assimilation. Another point to reconsider will be the simulation made with Feel++, as explained before, when we realized the simulation with the toolboxes available in Feel++, we had ignored the convective effects, so maybe we could try to do the simulation by introducing the convective effects. 
	

	
	\newpage
	
	\section*{Bibliography}
	\addcontentsline{toc}{section}{Bibliography}
	
	\printbibliography[heading=subbibintoc,keyword=ref,title={References}]
	
	\newpage
	
	\printbibliography[heading=subbibintoc,keyword=doc,title={Documentation}]
	
	\appendix
	
	\lstset{style=bash}

\section{Organisation of the repository}

	\begin{minipage}{0.65\linewidth}
		For the organisation of the repository, we created several directories (Figure \ref{tree}):
		\begin{enumerate}[label=\textbullet]
			\item \textbf{src :} In this directory there is all the source code of the project. We have a first \textit{python} subdirectory which contains the python codes for the data assimilation and parareal parts, these files were created during the M1 project. Then we have a second subdirectory \textit{cpp} which contains the C++ codes of the same 2 parts.
			\item \textbf{examples :} Here there are examples of how to use python and C++ codes.
			\item \textbf{tests :} In this directory there are python and C++ tests of the code. For python tests we will use the pytest tool and for C++ we will use ctest (see Appendix \ref{compile}).
			\item \textbf{docs :} This directory gathers all the files that are used to generate the documentation of the project (see Appendix \ref{doc}). The \textit{sphinx} and \textit{doxygen} directories enable respectively to document the python code and the C++ code (thanks to the sphinx~\cite{sphinx_doc} and doxygen~\cite{doxygen_doc} tools). The \textit{antora} directory contains some explanations of the project which are accessible online (the antora~\cite{antora_doc} tool was used). For example it explains : the differential equations concerned by this project, the data assimilation methods, the parareal method... The documentations generated by the previous tools are available directly in the GitHub repository via a Continuous Integration (CI) that has been set up (see Appendix \ref{ci}).
			The directories \textit{gantt}, \textit{meeting}, \textit{presentation} and \textit{report} contain all the latex files and images used for the presentations and reports requested in the context of the internship.
			\item \textbf{cmake/build :} The directory \textit{cmake} contains cmake configuration files which are used for the compilation of the C++ project (see Appendix \ref{compile}). The directory \textit{build} contains all the files generated by the compilation of the C++ code.
		\end{enumerate}
	\end{minipage} \qquad
	\begin{minipage}{0.33\linewidth}
		\begin{figure}[H]
			\includegraphics[width=\textwidth]{"images/appendix/tree.jpg"}
			\caption{Repository tree}
			\label{tree}
		\end{figure}
		%\lstinputlisting[language={},inputencoding=utf8]{tree.txt}
	\end{minipage}

\newpage

\section{Compile and test}
\label{compile}

	As explained in the previous section, the project includes both C++ and Python code. Let's see how to use these source codes :
	\begin{enumerate}[label=\textbullet]
		\item \textbf{Python :} \\
		Here all methods are grouped in a module called \textit{lorenz}. To execute the examples, we first need to add the path to the source code in the PYTHONPATH variable using the following command at the root of the project :
\begin{lstlisting}
		export PYTHONPATH=$PWD/src/python
\end{lstlisting}
		\fbox{ENKF ?} \\
		To run the examples for the parareal method with 2 processes, go to "examples/python/parareal" and then execute :
\begin{lstlisting}
		mkdir data_parareal
		mpirun -n 2 python3 examples_parareal.py [0,1,2]
\end{lstlisting}
		To run the tests we use the pytest tool. To do this, go to "tests/python" and run the following command :
\begin{lstlisting}
		pytest
\end{lstlisting}
		\item \textbf{C++ :} \\
		For the C++ code, we decided to write a CMake\footnote[1]{Minimum version required : 3.21} with presets~\cite{cmake_preset}.  A recurrent problem of CMake users is to share settings with other people for common ways of configuring a project. A solution to this is to define a "CMakePresets.json" file at the root of the project allowing to define different compilation modes. \\
		We defined 3 presets : \textit{default} is the default compilation mode, \textit{dbg} the mode to debug the code and \textit{doc} to generate the documentation (see Appendix \ref{doc}). \\
		These are the commands to configure and build in default mode :
\begin{lstlisting}
		cmake --preset default
		cmake --build --preset default
\end{lstlisting}
		The executables generated by the last command will then be stored in the "build/default/bin" directory. There you will find :
		\begin{itemize}[label=-]
			\item \textbf{enkf.e : } \fbox{ENKF ?}
			\item \textbf{parareal.e : } To run in parallel with 4 processes :
\begin{lstlisting}
		mpirun -n 4 ./build/default/bin/parareal.e
\end{lstlisting}
			This applies the parareal method to the Lorenz system with parameters defined in the code and displays the number of iterations of the method as well as the execution time. 
			\item \textbf{laplacian.e : } To run in sequential:
\begin{lstlisting}
		./build/default/bin/laplacian.e 
				--config-file <cfg_filename>
\end{lstlisting}
			To run in parallel with 4 processes:
\begin{lstlisting}
		mpirun -n 4 ./build/default/bin/laplacian.e 
				--config-file <cfg_filename>
\end{lstlisting}
			This allows to solve the Laplace equation from a given geometry by the finite element method by using Feel++~\cite{feelpp_laplacian}.
			\item \textbf{heat.e : } \\
			This solves in parallel the heat equation from a given geometry with the parareal method by using Feel++. \\
			We will place ourselves in the following test case. \\
			The spatial domain is partitioned into 2 sub-domains using the command :
\begin{lstlisting}
		feelpp_mesh_partitioner --dim=2 --part 2 
				--ifile <geo_filename> 
\end{lstlisting}
			We also partition the temporal domain into 2 sub-domains, which makes 2*2 processes for the fine integrator and 2 processes for the coarse integrator, so 6 processes in all. We can then run in parallel with 6 processes :
\begin{lstlisting}
		mpirun -np 6 ./build/default/bin/heat.e 
				--config-file <cfg_filename>
\end{lstlisting}		
		For more details see Section \ref{heat}.
		\end{itemize}
		For C++ tests, we use the ctest tool in the following way:
\begin{lstlisting}
		ctest --preset default
\end{lstlisting}
	\end{enumerate}

\newpage

\section{Documentation}
\label{doc}

	To document the project as well as possible, we have used different tools which we will talk about here. As said before, all the files concerning the documentation are in the \textit{docs} directory. \\
	Before presenting these tools, it is important to note that we have set up a Continuous Integration in order to automatically generate this documentation when a push is made on GitHub (see Section \ref{ci}). \\
	Here is the link to the documentation : \url{https://master-csmi.github.io/2022-m1-lorenz/}. \\
	To document the project we used 3 tools:
	\begin{itemize}[label=-]
		\item \textbf{Sphinx\cite{sphinx_doc}} : To document the Python code. 
		\item \textbf{Doxygen\cite{doxygen_doc}} : To document C++ code.
		\item \textbf{Antora\cite{antora_doc}} : To explain some parts/methods of the project (and to be able to keep it online).
	\end{itemize}
	We will now see how to generate the documentation locally using these 3 tools. For that, there are 2 methods : the first one is to generate "manually" the html files and the second one is to use the \textit{docs} preset of our CMake.
	\begin{enumerate}[label=\textbullet]
		\item \textbf{Manual generation :} 
		\begin{itemize}[label=-]
			\item \textbf{Antora :} Go to "docs/antora", then execute :
\begin{lstlisting}
		npm install
		npm run dev
		npm run serve
\end{lstlisting}
			Then click on the following link (in the "site-dev.yml" file) : \\ \url{http://127.0.0.1:8080/lorenz/1.0.0/index.html}. \\
			If we modify the documentation files, we just have to re-run the second line which will regenerate the documentation and then reload our browser window.
			\item \textbf{Sphinx :} Go to "docs/sphinx", then execute :
\begin{lstlisting}
		make html
		npm run sphinx
\end{lstlisting}
			\item \textbf{Doxygen :} Go to "docs/doxygen", then execute :
\begin{lstlisting}
		doxygen Doxyfile-dev.in
		npm run doxygen
\end{lstlisting}
		\end{itemize}
		\newpage
		\item \textbf{With CMake :} \\
		\danger This method is only available for Sphinx and Doxygen documentations. Indeed, CMake is configured in order that Antora documentation is generated from the files uploaded online on Github. This configuration is due to CI (see Section \ref{ci}). This means that local modifications will have no effect with this method. \\ \; \\
		Here is how to use the CMake \textit{docs} preset to generate the Sphinx and Doxygen documentations (by placing at the root of the project) :
\begin{lstlisting}
		cmake --preset docs
		cmake --build --preset docs
\end{lstlisting}
		Then in the same way as before, execute one of the following commands at the root of the project :
\begin{lstlisting}
		npm run sphinx
\end{lstlisting}
		or
\begin{lstlisting}
		npm run doxygen
\end{lstlisting}
	\end{enumerate}

\newpage

\section{Github actions}
\label{ci}

	As said before we have set up a CI (Continuous Integration) using GitHub actions.
	We added a badge at the beginning of the readme of our project to see when the build passed and when the build failed. Here is what it looks like in both cases : \\
	\begin{minipage}{0.48\linewidth}
		\begin{figure}[H]
			\centering
			\includegraphics[width=0.45\textwidth]{"images/appendix/badge_ok.jpg"}
			\caption{Build passing}
			\label{badge_ok}
		\end{figure}
	\end{minipage}
	\begin{minipage}{0.48\linewidth}
		\begin{figure}[H]
			\centering
			\includegraphics[width=0.5\textwidth]{"images/appendix/badge_not_ok.jpg"}
			\caption{Build failing}
			\label{badge_not_ok}
		\end{figure}
	\end{minipage} \\ \; \\
	For more details you have to go to the action tab of the repository where are detailed all the tasks of each commit. Here is a preview of the CI for one commit :
	\begin{figure}[H]
		\centering
		\includegraphics[width=0.65\textwidth]{"images/appendix/run_ci.jpg"}
		\caption{Tab actions in GitHub (for one commit)}
		\label{run_ci}
	\end{figure}
	\noindent Here are the jobs realized as soon as we push in the main branch :
	\begin{enumerate}[label=\textbullet]
		\item Compile C++ code using CMake default preset then use ctest to check if the tests pass.
		\item Test the Python code with pytest.
		\item Generate the documentation (Sphinx, Doxygen and Antora) using the docs preset of the CMake. Then, deploy the 3 documentations in the \textit{gh-pages} branch (in 3 respective directories: \textit{build\_sphinx}, \textit{build\_doxygen} and \textit{build\_antora)}. An html file has been created manually at the root of this branch in order to group the 3 documentations :
		\begin{figure}[H]
			\centering
			\includegraphics[width=0.25\textwidth]{"images/appendix/html_file.jpg"}
			\caption{Main page of the documentation}
			\label{html_file}
		\end{figure}
		For more details about the documentation see Section \ref{doc}.
	\end{enumerate}	
	
\end{document}
