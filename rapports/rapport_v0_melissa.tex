\documentclass[12pt]{article}
\usepackage{a4wide}
\usepackage{amsmath,amssymb}
\usepackage{bm}
\usepackage{enumitem}
\usepackage[colorlinks]{hyperref}
\newcommand{\vect}[1]{\hat{\boldsymbol{#1}}}

\begin{document}
	AYDOGDU Melissa \\
	LECOURTIER Frédérique \hfill 01 april 2022
	\begin{center}
		\Large\textbf{Project Report : Lorenz system}\\
	\end{center}
	
	\tableofcontents
	
	\section{Introduction}
	For centuries, humans have used math and science to make predictions about the universe. Scientist were able to predict an eclipse a century in advance. We are know able to make predictions about asteroids impact in earth. But why can we only predict the weather about a week or two in advance? Why scientists cannot predict the temperature for long periods of time? There have been many times when the weather predictions you checked the day before were completely different from the ones you expected.But it is important to know that weather prediction is a very complicated and difficult task to achieve. The complexity is strongly related to the earth's atmosphere, which can be considered as a fluid. 
	Around 1960 Edward Lorenz, an MIT meteorologist, started to work with a set of equations that represents a simplified version of the atmosphere, with 12 variables that’s changed over time.After two years of research , in 1963 he developed the Lorenz system, a simplified three-variable model to investigate atmospheric convection .This system defines a 3 dimensional trajectory by differential equations with 3 parameters.
	$$
	\begin{cases}
	    
        x'&=\sigma(y-x) \\
        y'&=x(r-z)-y \\
        z'&=xy-bz
       
    \end{cases}
    $$
    Here, $x$ is proportional to the rate of convection, $y$ is related to the horizontal temperature variation, and $z$ is the vertical temperature variation.
    We have also three parameters that represent:
    \begin{enumerate}[label=\textbullet]
		\item $\sigma$  relates to the Prandtl number
		\item $r$  relates to the Rayleigh number
		\item $b$ relates to the physical dimensions of the layer
	\end{enumerate}
	
   \noindent We can see that this system is non-linear , because in the second differential equation( $\frac{dy}{dt}$) we can see a $xz$ and in the third differential equation ($\frac{dz}{dt}$) we have a $xy$ . So the evolution of these three differential equations are all related.
   
   \noindent For the organization we have chosen to make a gant diagram with the different parts and the deadlines we have set (which can change during the time). The first part consists on understanding the system of Lorenz and to make the implementation of the resolution, since this part is essential for the rest, we worked together during this. For the remainder of the project, we divided the work, one person will take care of the para-reel part and the other on the part with the data assimilation.



	
	
	\section{Some interesting properties on the Lorenz System}
	\subsection{Theorie of chaos}
	In 1961, Lorenz was working with a simplified system that represented the atmosphere, with 12 variables. 
    They use a computer to calculated the values for each moments in time by applying the equations to the values from the previous moment. In some case, Lorenz’s team needed to re-start the simulation. Since the equations was the same, they expected to see the same results but sometimes they didn’t. At first, they think that the problem was coming from the computer ,but actually the error came from the fact that the computer was storing numbers with six decimal places but it was only printing the number with 3 decimal places, so the values that Lorenz put into the computer had tiny errors. So he discovered that in some systems, tiny changes in the initial conditions can lead to big changes in time.
    
	\noindent This effect was later named "butterfly effect". The "butterfly effect" represents a system that is very sensitive to the initial condition, which is why these systems are very difficult to predict over time. This idea gave birth to the notion of a butterfly flapping its wings in a region of the world and causing a tornado.
    This is why we call the Lorenz equations a chaotic system. This means that this type of system is roughly defined by sensitivity to initial conditions: infinitesimal differences in initial conditions of the system result in large differences in behavior.

	\subsection{ Présentations points fixes + déf attracteur}
	
	\section{First part : Numerical resolution with different methods}
	
	\begin{enumerate}[label=\textbullet]
		\item présentation du type de problème (généralisation et précision dans notre cas)
		\item discrétisation
		\item présentation des méthodes utilisées
		\item présentation d'au moins 1 résultat avec les différentes méthodes
	\end{enumerate}

\end{document}