\documentclass[12pt]{article}
\usepackage{a4wide}
\usepackage{amsmath,amssymb}
\usepackage{bm}
\usepackage{enumitem}
\usepackage[colorlinks]{hyperref}
\newcommand{\vect}[1]{\hat{\boldsymbol{#1}}}

\begin{document}
	AYDOGDU Melissa \\
	LECOURTIER Frédérique \hfill 01 april 2022
	\begin{center}
		\Large\textbf{Project Report : Lorenz system}\\
	\end{center}
	
	\tableofcontents
	
	\section{Introduction}
	
	\begin{enumerate}[label=\textbullet]
		\item exemples d'utilisation (météo...)
		\item présentation du système + propriétés mathématiques (non linéaire...) 
		\item but du projet, "répartition" du travail, diagramme de gantt (en annexe)
	\end{enumerate}
	
	\section{Some interesting properties on the Lorenz System}
	
	\begin{enumerate}[label=\textbullet]
		\item Théorie du chaos (définition + comment lorenz l'a découvert)
		\item Présentations points fixes + déf attracteur
	\end{enumerate}
	
	\section{First part : Numerical resolution with different methods}
	
	\begin{enumerate}[label=\textbullet]
		\item présentation du type de problème (généralisation et précision dans notre cas)
		\item discrétisation
		\item présentation des méthodes utilisées
		\item présentation d'au moins 1 résultat avec les différentes méthodes
	\end{enumerate}

\end{document}