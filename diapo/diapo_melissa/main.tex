%% Requires compilation with XeLaTeX or LuaLaTeX
\documentclass[10pt,xcolor={table,dvipsnames},t]{beamer}
\usetheme{UCBerkeley}
\usepackage{amsmath}

\usepackage{graphicx}
\usepackage{hyperref}


\title[Your Short Title]{Lorenz System Project}
\subtitle{Presentation (version 0)}
\author[name]{AYDOGDU Melissa, LECOURTIER Frédérique}
\institute{\large Strasbourg University}
\date{05 april 2022}

\useoutertheme{miniframes}

\begin{document}

\begin{frame}
  \titlepage
\end{frame}

\AtBeginSection[]{
  \begin{frame}
  \vfill
  \centering
  \begin{beamercolorbox}[sep=5pt,shadow=true,rounded=true]{subtitle}
    \usebeamerfont{title}\insertsectionhead\par%
  \end{beamercolorbox}
  \vfill
  \end{frame}
}



\section{Introduction}


\begin{frame}{Lorenz system}
    
    The system :
    $$\left\{\begin{aligned} 
    x'&=\sigma(y-x) \\
    y'&=x(r-z)-y \\
    z'&=xy-bz
    \end{aligned}\right.$$
    
    \begin{enumerate}
		\item Non-lineary- two nonlinearities: $xy$ and $xz$
		\item symmetry - invarient under $(x,y)\rightarrow(-x,-y) $
		\item 3 fixed points- $(0,0,0) ,(\sqrt{b(r-1)},\sqrt{b(r-1)},r-1)$ and $(-\sqrt{b(r-1)},-\sqrt{b(r-1)},r-1)$
	\end{enumerate}

\end{frame}
\begin{frame}{Goals of the project}
    
    \pgfimage[height=4.5cm,width=10.0cm]{gantt}

\end{frame}


\section{Some interesting properties}
\begin{frame}{Chaos theory}
    \pgfimage[height=4.0cm,width=10.0cm]{chaos1}
\end{frame}
\begin{frame}
     \begin{center}
        \pgfimage[height=4cm,width=10.0cm]{chaos2}
    \end{center}
\end{frame}
\begin{frame}{Fixed points}

    Some examples of commonly used commands and features are included, to help you get started.
    \begin{itemize}
      \item Your introduction goes here!
      \item Use \texttt{itemize} to organize your main points.
    \end{itemize}

\end{frame}

\section{Numerical resolution with different methods}

\begin{frame}{Methods used to solve Lorenz system}

    Some examples of commonly used commands and features are included, to help you get started.
    \begin{itemize}
      \item Your introduction goes here!
      \item Use \texttt{itemize} to organize your main points.
    \end{itemize}

\end{frame}

\begin{frame}{Some results (with Python)}

    Some examples of commonly used commands and features are included, to help you get started.
    \begin{itemize}
      \item Your introduction goes here!
      \item Use \texttt{itemize} to organize your main points.
    \end{itemize}
    
\end{frame}

\end{document}
