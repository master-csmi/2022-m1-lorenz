\documentclass[12pt]{article}
\usepackage{a4wide}
\usepackage{amsmath,amssymb}
\usepackage{bm}
\usepackage{enumitem}
\usepackage[colorlinks]{hyperref}
\usepackage{graphicx}
\usepackage{appendix}
\newcommand{\vect}[1]{\hat{\boldsymbol{#1}}}

\begin{document}
	\nocite{*}
	AYDOGDU Melissa \\
	LECOURTIER Frédérique \hfill 01 april 2022
	\begin{center}
		\Large\textbf{Project Report : Lorenz system}\\
	\end{center}
	
	\tableofcontents
	
	\newpage
	
	\section{Introduction}
	
	For centuries, humans have used math and science to make predictions about the universe. Scientist were able to predict an eclipse a century in advance. We are know able to make predictions about asteroids impact in earth. But why can we only predict the weather about a week or two in advance? Why scientists cannot predict the temperature for long periods of time? There have been many times when the weather predictions you checked the day before were completely different from the ones you expected.But it is important to know that weather prediction is a very complicated and difficult task to achieve. The complexity is strongly related to the earth's atmosphere, which can be considered as a fluid. 
	Around 1960 Edward Lorenz, an MIT meteorologist, started to work with a set of equations that represents a simplified version of the atmosphere, with 12 variables that’s changed over time.After two years of research , in 1963 he developed the Lorenz system, a simplified three-variable model to investigate atmospheric convection .This system defines a 3 dimensional trajectory by differential equations with 3 parameters.
	$$
	\begin{cases}
		
		x'&=\sigma(y-x) \\
		y'&=x(r-z)-y \\
		z'&=xy-bz
		
	\end{cases}
	$$
	Here, $x$ is proportional to the rate of convection, $y$ is related to the horizontal temperature variation, and $z$ is the vertical temperature variation.
	We have also three parameters that represent:
	\begin{enumerate}[label=\textbullet]
		\item $\sigma$  relates to the Prandtl number
		\item $r$  relates to the Rayleigh number
		\item $b$ relates to the physical dimensions of the layer
	\end{enumerate}
	
	\noindent We can see that this system is non-linear , because in the second differential equation( $\frac{dy}{dt}$) we can see a $xz$ and in the third differential equation ($\frac{dz}{dt}$) we have a $xy$ . So the evolution of these three differential equations are all related. \\
	
	\noindent Let us now determine the fixed points of the Lorenz system. These are the points such that $X'=0$. 
	
	$$
	\left\{\begin{aligned}
		x'=&\sigma(y-x) &&=0 \\
		y'=&x(r-z)-y  &&=0\\
		z'=&xy-bz &&=0
	\end{aligned}\right. 
	\quad \Rightarrow \quad 
	\left\{\begin{aligned}
		&x=y \\
		&(r-1-z)x=0\\
		&x^2=bz
	\end{aligned}\right.
	$$
	
	\begin{enumerate}[label=\textbullet]
		\item If $x=0$ : \quad  $y=0$ and $z=0$.
		\item If $x\ne 0$ and $r>1$ : \quad $\left\{\begin{aligned}
			&z=r-1\\
			&x=y=\pm\sqrt{b(r-1)}
		\end{aligned}\right.
		$
	\end{enumerate}
	
	\noindent We deduce that the fixed points of the Lorenz system are : $(0,0,0)$ for all values of the parameters. And for $r>1$, there is also a pair of fixed points $(\sqrt{b(r-1)},\sqrt{b(r-1)},r-1)$ and $(-\sqrt{b(r-1)},-\sqrt{b(r-1)},r-1)$. \\
	
	\noindent For the organization we have chosen to make a gantt diagram (section~\ref{diag}) with the different parts and the deadlines we have set (which can change during the time). The first part consists on understanding the system of Lorenz and to make the implementation of the resolution, since this part is essential for the rest, we worked together during this. For the remainder of the project, we divided the work, one person will take care of the para-reel part and the other on the part with the data assimilation.
	
	\section{Some interesting properties on the Lorenz System}
	
	\subsection{Lorenz Attractor}
	
	The Lorenz system is also called Lorenz Attractor. Let's try to undestand why. Lorenz discovered that an incredible structure emerges when $x(t)$ is plotted against $z(t)$. The famous butterfly wing pattern appears :
	\begin{center}
		\includegraphics[width=0.5\textwidth]{"images/butterfly.jpg"}
	\end{center}
	\begin{enumerate}[label=\textbullet]
		\item The trajectory seems to cross several times, but this is just an illusion of the projection of the trajectory in three dimensions on a plane in two dimensions. In 3D, there are no crossings!
		\item The number of circuits made on either side varies unpredictably from one cycle to the next. The sequence of the number of circuits in each lobe has many of the characteristics of a random sequence!
		\item When the trajectory is viewed in 3 dimensions, it appears to settle onto a thin set that looks like a pair of butterfly wings. We call this attractor a strange attractor and it can be shown schematically as :
		\begin{center}
			\includegraphics[width=0.5\textwidth]{"images/butterfly3D.jpg"}
		\end{center}
	\end{enumerate}
	
	\noindent The uniqueness theorem means that trajectories cannot cross or merge, hence the two surfaces of the strange attractor can only appear to merge. Lorenz concluded that “there is an infinite
	complex of surfaces” where they appear to merge. Today this “infinite complex of surfaces” would be called a fractal which is a set of points with zero volume but infinite surface area.
	
	\noindent The term attractor is also difficult to define in a rigorous way. Loosely, an attractor is a set of points to which all neighbouring trajectories converge. Stable fixed points and
	stable limit cycles are examples. Nobody has yet proved that the Lorenz attractor is truly an attractor. Finally we define a strange attractor to be an attractor that exhibits sensitive dependence
	on initial conditions.
	
	\subsection{Theory of chaos}
	
	In 1961, Lorenz was working with a simplified system that represented the atmosphere, with 12 variables. 
	They use a computer to calculated the values for each moments in time by applying the equations to the values from the previous moment. In some case, Lorenz’s team needed to re-start the simulation. Since the equations was the same, they expected to see the same results but sometimes they didn’t. At first, they think that the problem was coming from the computer ,but actually the error came from the fact that the computer was storing numbers with six decimal places but it was only printing the number with 3 decimal places, so the values that Lorenz put into the computer had tiny errors. So he discovered that in some systems, tiny changes in the initial conditions can lead to big changes in time.
	
	\noindent This effect was later named "butterfly effect". The "butterfly effect" represents a system that is very sensitive to the initial condition, which is why these systems are very difficult to predict over time. This idea gave birth to the notion of a butterfly flapping its wings in a region of the world and causing a tornado.
	This is why we call the Lorenz equations a chaotic system. This means that this type of system is roughly defined by sensitivity to initial conditions: infinitesimal differences in initial conditions of the system result in large differences in behavior.
	
	\section{First part : Numerical resolution with different methods}
	
	We consider $f : [0; T] \times \mathbb{R}^n \rightarrow \mathbb{R}^n$ a continuous function. For $X_0\in \mathbb{R}^n$, the problem is to find  $X\in C^1([0,T],\mathbb{R}^n)$ a solution for the differential equation:
	
	$$\left\{\begin{aligned}
		X'&=f(t,X) \\
		X(0)&=X_0
	\end{aligned}\right.$$
	
	\noindent To solve the Lorenz problem we will have:
	
	$$X'=\begin{pmatrix}
		x' \\
		y' \\
		z'
	\end{pmatrix}, \quad X=\begin{pmatrix}
		x \\
		y \\
		z
	\end{pmatrix} \quad et \quad f(t,X)=\begin{pmatrix}
		\sigma(y-x) \\
		x(r-z)-y \\
		xy-bz
	\end{pmatrix}$$
	
	\noindent After discretizing the problem in time, we can implement different methods of solving the ODE. The methods are the following and will be described in more detail in the following parts : explicit Euler, implicit Euler and Runge Kutta (order 4). We will also use a scipy function and finally compare all our methods.
	
	\subsection{Discretization}
	
	To solve the problem, we will use the finite difference method. We will start by slicing the interval $[0,T]$ in $N+1$ discretization points (so $N$ intervals). Let $t_n=n\Delta t$ the discretization timed with $\Delta t=T/N$ the time step. Then, we denote by $X_n=X(t_n)$ the discretization points. So for $n=\{0,\dots,N\}$, we will have:
	
	$$\left\{\begin{aligned} 
		x_n&=x(t_n)=x(n\Delta t) \\
		y_n&=y(t_n)=y(n\Delta t) \\
		z_n&=z(t_n)=z(n\Delta t)
	\end{aligned}\right.$$
	
	\noindent By Taylor's theorem, we get :
	
	$$\begin{aligned}
		&&X(t+\Delta t)&=X(t)+\Delta t X'(t) + O(\Delta t^2) \\
		\Rightarrow&& \quad X'(t)&=\frac{X(t+\Delta t)-X(t)}{\Delta t} + O(\Delta t) \\
		\Rightarrow&& \quad \partial_t X_n&\approx\frac{X_{n+1}-X_n}{\Delta t} \\
	\end{aligned}
	$$	
	
	\subsection{Explicit Euler}
	
	The explicit Euler method is written :
	
	$$X_{n+1}=X_n+\Delta t f(t_n,X_n)$$
	
	\noindent which will give us:
	
	$$\left\{\begin{aligned} 
		x_{n+1}&=\sigma\Delta t y_n+(1-\sigma\Delta t) x_n \\
		y_{n+1}&=\Delta t x_n(r-z_n)(1-\Delta t)y_n \\
		z_{n+1}&=\Delta t x_ny_n+(1-b\Delta t)z_n
	\end{aligned}\right.$$
	
	\subsection{Implicit Euler}
	
	The implicit Euler method is written :
	
	$$X_{n+1}=X_n+\Delta t f(t_{n+1},X_{n+1})$$
	
	\noindent which will give us:
	
	$$\left\{\begin{aligned} 
		x_{n+1}&=x_n+\Delta t\sigma(y_{n+1}-x_{n+1}) \\
		y_{n+1}&=y_n+\Delta t x_{n+1}(r-z_{n+1})-y_{n+1} \\
		z_{n+1}&=z_n+\Delta tx_{n+1}y_{n+1}-\Delta tbz_{n+1}
	\end{aligned}\right.$$
	
	\begin{enumerate}[label=\textbullet]
		\item First, we isolate the $n+1$ terms on the left and the $n$ terms on the right:
		
		$$\left\{\begin{aligned} 
			(1+\Delta t\sigma)x_{n+1}-\Delta t\sigma y_{n+1}&=x_n \\
			-\Delta t(r-z_{n+1})x_{n+1}+(1+\Delta t)y_{n+1}&=y_n \\
			-\Delta ty_{n+1}x_{n+1}+(1+\Delta tb)z_{n+1}&=z_n
		\end{aligned}\right.$$
		
		\item We can then linearize the terms 
		
		$$\begin{aligned}
			x_{n+1}y_{n+1}&\approx x_{n+1,k+1}y_{n+1,k} \\
			x_{n+1}z_{n+1}&\approx x_{n+1,k+1}z_{n+1,k}
		\end{aligned}$$ 		
		\noindent Then, we get :
		
		$$\left\{\begin{aligned} 
			(1+\Delta t\sigma)x_{n+1,k+1}-\Delta t\sigma y_{n+1,k+1}&=x_n \\
			-\Delta t(r-z_{n+1,k})x_{n+1,k+1}+(1+\Delta t)y_{n+1,k+1}&=y_n \\
			-\Delta ty_{n+1,k}x_{n+1,k+1}+(1+\Delta tb)z_{n+1,k+1}&=z_n
		\end{aligned}\right.$$
		
		\item We can then put in matrix form  $M(X_{n+1,k})X_{n+1,k+1}=X_n$ with :
		
		$$M(X_{n+1,k})=\begin{pmatrix}
			1+\Delta t\sigma & -\Delta t\sigma & 0 \\
			-\Delta t(r-z_{n+1,k}) & 1+\Delta t & 0 \\
			-\Delta ty_{n+1,k} & 0 & 1+\Delta tb
		\end{pmatrix},$$
		$$X_{n+1,k+1}=\begin{pmatrix}
			x_{n+1,k+1} \\
			y_{n+1,k+1} \\
			z_{n+1,k+1}
		\end{pmatrix} \quad et \quad X_n=\begin{pmatrix}
			x_n \\
			y_n \\
			z_n
		\end{pmatrix} $$
	\end{enumerate}
	
	\subsection{Runge Kutta}
	
	\begin{enumerate}[label=\textbullet]
		\item \textbf{Runge Kutta (order 4th) :} \\
		The Runge Kutta method of order 4 is written :
		
		$$X_{n+1}=X_n+\frac{\Delta t}{6}\left(K_1+2K_2+2K_3+K_4\right)$$
		
		\noindent where 
		
		$$\left\{\begin{aligned}
			K_1&=f(t_n,X_n) \\
			K_2&=f\left(t_n+\frac{\Delta t}{2},X_n+\frac{1}{2} K_1\Delta t\right) \\
			K_3&=f\left(t_n+\frac{\Delta t}{2},X_n+\frac{1}{2} K_2\Delta t\right) \\
			K_4&=f\left(t_n+\Delta t,X_n+K_3\Delta t\right)
		\end{aligned}\right.$$
		\item \textbf{Scipy function :} \href{https://docs.scipy.org/doc/scipy/reference/generated/scipy.integrate.solve_ivp.html#scipy.integrate.solve_ivp}{scipy.integrate.solve\_ivp}
	\end{enumerate}
	
	\subsection{Comparing methods}
	
	\begin{enumerate}[label=\textbullet]
		\item First, we had to implement a function allowing to plot a 2D graph representing x versus y, then a 2D graph representing x versus z and finally a 3D graph of the solution. Then, we implemented the different methods described above and we checked visually that they worked correctly. Here are the graphs obtained with the following parameters :
		\begin{center}
			$\sigma=10,\quad \beta=8/3 \quad r=9/10$ \\
			$X_0=(-10,10,5)$ \\
			$N=5000, \quad T=100$
		\end{center}
		\includegraphics[width=\textwidth]{"images/euler_explicit.png"}
		\includegraphics[width=\textwidth]{"images/euler_explicit_dt2.png"}
		\includegraphics[width=\textwidth]{"images/euler_implicit.png"}
		\includegraphics[width=\textwidth]{"images/RK4_Lorenz.png"}
		\includegraphics[width=\textwidth]{"images/scipy.png"}
		\item Then, we compared the execution times of these different methods. Here is what we get with the same parameters : \\
		\includegraphics[width=0.7\textwidth]{"images/execution_times.jpg"} \\
		It seems that the explicit Euler method is the fastest, followed by the scipy function.
	\end{enumerate}
		
	\newpage
	
	\section{Gantt chart}
	\label{diag}
	
	\centering
	\includegraphics[angle=90,width=0.78\textwidth]{gantt.pdf}
	
	\newpage	
	\bibliographystyle{plain}
	\bibliography{biblio}
	
\end{document}