%% Requires compilation with XeLaTeX or LuaLaTeX
\documentclass[10pt,xcolor={table,dvipsnames},t]{beamer}
%\documentclass[compress]{beamer}
\usetheme{diapo}
\usepackage{amsmath}
\usepackage[bottom]{footmisc}
\usepackage{multirow}
\usepackage[francais]{babel}

\title[Lorenz]{Application de la méthode para-real à la simulation d'EDP dans Feel++}
\subtitle{Présentation 2}
\author[name]{LECOURTIER Frédérique}
\institute{\large Université de Strasbourg}
\date{\today}

\begin{document}
	
	\begin{frame}
		\titlepage
	\end{frame}
	
	\AtBeginSection[]{
		\begin{frame}
			\vfill
			\centering
			\begin{beamercolorbox}[sep=5pt,shadow=true,rounded=true]{subtitle}
				\usebeamerfont{title}\insertsectionhead\par%
			\end{beamercolorbox}
			\vfill
		\end{frame}
	}
	
	\begin{frame}{Fait la semaine dernière}
		\begin{enumerate}[\textbullet]
			\item Méthode para-real en C++ fonctionne dans différents cas (comparaison avec les résultats python).
			\item Organisation du repository, création d'un cmake et utilisation de l'interface de vscode.
			\item Documentation du code et utilisation de doxygen. \\ \;
			\item Workflow github pour générer automatiquement la documentation du code (non abouti).
			\item Workflow github pour faire les tests en utilisant ctest (non abouti).
		\end{enumerate}
	\end{frame}
	
	\begin{frame}{Étapes pour cette semaine}
		\begin{enumerate}[\textbullet]
			\item Faire fonctionner les actions github.
			\item Faire des tests pour vérifier le code. \\ \;
			\item Implémenter la résolution de l'équation de Laplace en C++ avec Feel++.
			\item Commencer à implémenter la résolution de l'équation de la chaleur en C++ avec Feel++.
		\end{enumerate}
	\end{frame}
	
	
\end{document}