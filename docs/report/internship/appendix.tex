\section{Organisation of the repository}

	\begin{minipage}{0.65\linewidth}
		For the organisation of the repository, we created several directories (Figure \ref{tree}):
		\begin{enumerate}[label=\textbullet]
			\item \textbf{src :} In this directory there is all the source code of the project. We have a first \textit{python} subdirectory which contains the python codes for the data assimilation and parareal parts, these files were created during the M1 project. Then we have a second subdirectory \textit{cpp} which contains the C++ codes of the 2 parts.
			\item \textbf{examples :} Here there are examples of the use of python and C++ codes.
			\item \textbf{tests :} In this directory there are python and C++ tests of the code. For python tests we will use the pytest tool and for C++ we will use ctest (see Section \ref{compile}).
			\item \textbf{docs :} This directory gathers all the files that allow the documentation of the project (see section \ref{doc}). The \textit{sphinx} and \textit{doxygen} directories allow respectively to document the python code and the C++ code (thanks to the sphinx and doxygen tools). The \textit{antora} directory contains a kind of report of the project accessible on line (the antora tool was used) and allows to explain the differential equations concerned by this project, the data assimilation methods, the parareal method... The documentations generated by the previous tools are available directly in the github repository via a CI that has been set up (see section \ref{ci}).
			The directories \textit{gantt}, \textit{meeting}, \textit{presentation} and \textit{report} contain all the latex files and images used for the presentations and reports requested in the context of the internship.
		\end{enumerate}
	\end{minipage} \qquad
	\begin{minipage}{0.33\linewidth}
		\begin{figure}[H]
			\includegraphics[width=\textwidth]{"images/appendix/tree.jpg"}
			\caption{Repository tree}
			\label{tree}
		\end{figure}
		%\lstinputlisting[language={},inputencoding=utf8]{tree.txt}
	\end{minipage}

\section{Compile and test}
\label{compile}

\section{Documentation}
\label{doc}

\section{Github actions}
\label{ci}