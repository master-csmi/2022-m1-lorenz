\lstset{style=bash}

\section{Organisation of the repository}

	\begin{minipage}{0.65\linewidth}
		For the organisation of the repository, we created several directories (Figure \ref{tree}):
		\begin{enumerate}[label=\textbullet]
			\item \textbf{src :} In this directory there is all the source code of the project. We have a first \textit{python} subdirectory which contains the python codes for the data assimilation and parareal parts, these files were created during the M1 project. Then we have a second subdirectory \textit{cpp} which contains the C++ codes of the same 2 parts.
			\item \textbf{examples :} Here there are examples of how to use python and C++ codes.
			\item \textbf{tests :} In this directory there are python and C++ tests of the code. For python tests we will use the pytest tool and for C++ we will use ctest (see Appendix \ref{compile}).
			\item \textbf{docs :} This directory gathers all the files that are used to generate the documentation of the project (see Appendix \ref{doc}). The \textit{sphinx} and \textit{doxygen} directories enable respectively to document the python code and the C++ code (thanks to the sphinx~\cite{sphinx_doc} and doxygen~\cite{doxygen_doc} tools). The \textit{antora} directory contains some explanations of the project which are accessible online (the antora~\cite{antora_doc} tool was used). For example it explains : the differential equations concerned by this project, the data assimilation methods, the parareal method... The documentations generated by the previous tools are available directly in the GitHub repository via a Continuous Integration (CI) that has been set up (see Appendix \ref{ci}).
			The directories \textit{gantt}, \textit{meeting}, \textit{presentation} and \textit{report} contain all the latex files and images used for the presentations and reports requested in the context of the internship.
			\item \textbf{cmake/build :} The directory \textit{cmake} contains cmake configuration files which are used for the compilation of the C++ project (see Appendix \ref{compile}). The directory \textit{build} contains all the files generated by the compilation of the C++ code.
		\end{enumerate}
	\end{minipage} \qquad
	\begin{minipage}{0.33\linewidth}
		\begin{figure}[H]
			\includegraphics[width=\textwidth]{"images/appendix/tree.jpg"}
			\caption{Repository tree}
			\label{tree}
		\end{figure}
		%\lstinputlisting[language={},inputencoding=utf8]{tree.txt}
	\end{minipage}

\newpage

\section{Compile and test}
\label{compile}

	As explained in the previous section, the project includes both C++ and Python code. Let's see how to use these source codes :
	\begin{enumerate}[label=\textbullet]
		\item \textbf{Python :} \\
		Here all methods are grouped in a module called \textit{lorenz}. To execute the examples, we first need to add the path to the source code in the PYTHONPATH variable using the following command at the root of the project :
\begin{lstlisting}
		export PYTHONPATH=$PWD/src/python
\end{lstlisting}
		\fbox{ENKF ?} \\
		To run the examples for the parareal method with 2 processes, go to "examples/python/parareal" and then execute :
\begin{lstlisting}
		mkdir data_parareal
		mpirun -n 2 python3 examples_parareal.py [0,1,2]
\end{lstlisting}
		To run the tests we use the pytest tool. To do this, go to "tests/python" and run the following command :
\begin{lstlisting}
		pytest
\end{lstlisting}
		\item \textbf{C++ :} \\
		For the C++ code, we decided to write a CMake\footnote[1]{Minimum version required : 3.21} with presets~\cite{cmake_preset}.  A recurrent problem of CMake users is to share settings with other people for common ways of configuring a project. A solution to this is to define a "CMakePresets.json" file at the root of the project allowing to define different compilation modes. \\
		We defined 3 presets : \textit{default} is the default compilation mode, \textit{dbg} the mode to debug the code and \textit{doc} to generate the documentation (see Appendix \ref{doc}). \\
		These are the commands to configure and build in default mode :
\begin{lstlisting}
		cmake --preset default
		cmake --build --preset default
\end{lstlisting}
		The executables generated by the last command will then be stored in the "build/default/bin" directory. There you will find :
		\begin{itemize}[label=-]
			\item \textbf{enkf.e : } \fbox{ENKF ?}
			\item \textbf{parareal.e : } To run in parallel with 4 processes :
\begin{lstlisting}
		mpirun -n 4 ./build/default/bin/parareal.e
\end{lstlisting}
			This applies the parareal method to the Lorenz system with parameters defined in the code and displays the number of iterations of the method as well as the execution time. 
			\item \textbf{laplacian.e : } To run in sequential:
\begin{lstlisting}
		./build/default/bin/laplacian.e 
				--config-file <cfg_filename>
\end{lstlisting}
			To run in parallel with 4 processes:
\begin{lstlisting}
		mpirun -n 4 ./build/default/bin/laplacian.e 
				--config-file <cfg_filename>
\end{lstlisting}
			This allows to solve the Laplace equation from a given geometry by the finite element method by using Feel++~\cite{feelpp_laplacian}.
			\item \textbf{heat.e : } \\
			This solves in parallel the heat equation from a given geometry with the parareal method by using Feel++. \\
			We will place ourselves in the following test case. \\
			The spatial domain is partitioned into 2 sub-domains using the command :
\begin{lstlisting}
		feelpp_mesh_partitioner --dim=2 --part 2 
				--ifile <geo_filename> 
\end{lstlisting}
			We also partition the temporal domain into 2 sub-domains, which makes 2*2 processes for the fine integrator and 2 processes for the coarse integrator, so 6 processes in all. We can then run in parallel with 6 processes :
\begin{lstlisting}
		mpirun -np 6 ./build/default/bin/heat.e 
				--config-file <cfg_filename>
\end{lstlisting}		
		For more details see Section \ref{heat}.
		\end{itemize}
		For C++ tests, we use the ctest tool in the following way:
\begin{lstlisting}
		ctest --preset default
\end{lstlisting}
	\end{enumerate}

\newpage

\section{Documentation}
\label{doc}

\newpage

\section{Github actions}
\label{ci}