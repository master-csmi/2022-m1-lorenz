%% Requires compilation with XeLaTeX or LuaLaTeX
\documentclass[10pt,xcolor={table,dvipsnames},t]{beamer}
\usetheme{diapo}
\usepackage{amsmath}

\title[Your Short Title]{Lorenz System Project}
\subtitle{Presentation (version 0)}
\author[name]{AYDOGDU Melissa, LECOURTIER Frédérique}
\institute{\large Strasbourg University}
\date{05 april 2022}

\useoutertheme{miniframes}

\begin{document}
	
	\begin{frame}
		\titlepage
	\end{frame}
	
	\AtBeginSection[]{
		\begin{frame}
			\vfill
			\centering
			\begin{beamercolorbox}[sep=5pt,shadow=true,rounded=true]{subtitle}
				\usebeamerfont{title}\insertsectionhead\par%
			\end{beamercolorbox}
			\vfill
		\end{frame}
	}

	\section{Goals of the project}

	\begin{frame}{Cemosis}
		
		\begin{minipage}{0.4\hsize}
			\pgfimage[width=\linewidth]{images/logo-cemosis.pdf}
		\end{minipage} \quad
		\begin{minipage}{0.5\hsize}
			\begin{enumerate}[\textbullet]
				\item \textbf{MSO} \\
				Modeling Simulation and Optimization
				\item \textbf{DS} \\
				Data Science, Big Data, Smart Data
				\item \textbf{HPC} \\
				High Performance Computing, \\
				Parallel Computing, Cloud Computing
				\item \textbf{SI} \\
				Signal and Image processing
			\end{enumerate}
		\end{minipage}
		
	\end{frame}

	\begin{frame}{Objectives}
		
		 \begin{enumerate}[\textbullet]
			\item \textbf{para-real method} - \quad to accelerate the simulation of ODE
			\item \textbf{data assimilation}, specially
			\begin{itemize}
				\item understanding Kalman Filter
				\item understanding Ensemble Kalman Filter
			\end{itemize}
		\end{enumerate}	 	
		
	\end{frame}
	
	\begin{frame}{Gantt chart}
		
		\pgfimage[height=5.5cm,width=11cm]{images/gantt}
		
	\end{frame}
	
	\section{Some interesting properties}
	
	\begin{frame}[allowframebreaks]{Lorenz system}
		
		The system :
		$$\left\{\begin{aligned} 
			x'&=\sigma(y-x) \\
			y'&=x(r-z)-y \\
			z'&=xy-bz
		\end{aligned}\right.$$
	
		where
		
		\begin{enumerate}[\textbullet]
			\item $\sigma > 0$  relates to the Prandtl number
			\item $r > 0$  relates to the Rayleigh
			\item $b > 0$ is a geometric factor
		\end{enumerate}
	
		\newpage 
		
		Some interesting properties of the system :
		
		\begin{enumerate}[\textbullet]
			\item Non-linear - two nonlinearities : $xy$ and $xz$
			\item symmetry - invarient under $(x,y)\rightarrow(-x,-y) $
			\item 3 fixed points - $(0,0,0) ,(\sqrt{b(r-1)},\sqrt{b(r-1)},r-1)$ and $(-\sqrt{b(r-1)},-\sqrt{b(r-1)},r-1)$
		\end{enumerate}
		
	\end{frame}
	
	\begin{frame}{Lorenz Attractor}
		
		\begin{figure}[h]
			\begin{minipage}[c]{.46\linewidth}
				\centering
				\includegraphics[width=0.8\textwidth]{images/butterfly.jpg}
				\caption{Butterfly wing pattern - in 2D}
			\end{minipage}
			\hfill
			\begin{minipage}[c]{.46\linewidth}
				\centering
				\includegraphics[width=\textwidth]{images/butterfly3D.jpg}
				\caption{Strange attractor - in 3D \qquad\qquad
					(THREE DIMENSIONAL SYSTEMS, \qquad\qquad
					Lecture 6 : The Lorenz Equations)}
			\end{minipage}
		\end{figure}
		
	\end{frame}
	
	\begin{frame}{Chaos theory}
		
		\begin{center}
			\begin{minipage}[c]{.99\linewidth}
				\centering
				\includegraphics[width=\linewidth]{images/chaos.png}
			\end{minipage}
		\end{center}
		
	\end{frame}
	
	\section{Numerical resolution with different methods}
	
	\begin{frame}{Methods used to solve Lorenz system}
		
		\begin{itemize}
			\item Discretization : \\
			\qquad \includegraphics[width=0.4\textwidth]{images/discretization.jpg} \\
			\item Explicit Euler
			\item Implicit Euler			
			\item Runge Kutta order 4th			
			\item Scipy function : \qquad \textit{scipy.integrate.solve\_ivp}
		\end{itemize}
		
	\end{frame}
	
	\begin{frame}{Some results (with Python)}
		
		\includegraphics[width=\textwidth]{images/N100000.png} \\ 
		\begin{center}
			\begin{minipage}[c]{0.5\linewidth}
				$\sigma=10,\quad \beta=8/3, \quad r=28$ \\
				$X_0=(-10,10,5)$ 
			\end{minipage}
			$T=100, N = 100000 \Rightarrow \Delta t=0.001$
		\end{center}
		
	\end{frame}
	
\end{document}

